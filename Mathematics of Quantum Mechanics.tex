\documentclass[11pt]{article}
\renewcommand{\baselinestretch}{1.05}

\usepackage{amsmath,amsthm,verbatim,amssymb,amsfonts,amscd, graphicx}
\usepackage{graphics}

\usepackage{xcolor}

\usepackage[hidelinks]{hyperref}
\usepackage{parskip}

\usepackage{subcaption}
\usepackage{wrapfig}
\usepackage{multirow}

\setlength\tabcolsep{1mm}
\renewcommand\arraystretch{1.3}

\setlength\voffset{-1in}
\setlength\hoffset{-1in}
\setlength\topmargin{0.7874in}
\setlength\oddsidemargin{0.7874in}
\setlength\textheight{9.518099in}
\setlength\textwidth{6.6932993in}
\setlength\footskip{0.0cm}
\setlength\headheight{0cm}
\setlength\headsep{0cm}

\newcommand\textstyleEmphasis[1]{\textit{#1}}

\usepackage{braket}

\newtheorem{theorem}{Theorem}[section]
\newtheorem{corollary}{Corollary}[theorem]
\newtheorem{lemma}[theorem]{Lemma}

\theoremstyle{definition}
\newtheorem{definition}{Definition}[section]

\numberwithin{equation}{section}

\begin{document}

\title{\textbf{Mathematics of Quantum Mechanics} \\ \textbf{\textcolor{red}{Draft version}} }
\author{Mohamed Hage Hassan}
\date{8 May 2017}
\maketitle
\thispagestyle{empty}

\begin{abstract}
     
\end{abstract}

\vskip 9.5cm
\begin{center} \textbf{Institut Polytechnique de Grenoble} \end{center}

\clearpage
\tableofcontents

\clearpage

\section*{Introduction}
\addcontentsline{toc}{section}{Introduction}

This text is largely inspired from professor Ramamurti Shankar's Principles of Quantum Mechanics\cite{Principles of Quantum Mechanics}.

\section{Linear Vector Spaces}

Let's recall the basic definition of a vector space : 
\medskip
\iffalse
\begin{theorem}
Let $f$ be a function whose derivative exists in every point, then $f$ is 
a continuous function.
\end{theorem}
\fi

\theoremstyle{definition}
\begin{definition}{\textbf{($\kappa$-vectorial space)}}\\
  Let $X$ be a non empty set, which forms a $\kappa$-vectorial (linear) space, having :
  \begin{itemize} \itemsep -4pt
  \item An inner product :
    \begin{align*}
    f : X \times X \to X \\
    (u, v) \to u + v
    \end{align*}
  \item An extern product :
    \begin{align*}
    \kappa \times X \to X \\ 
    (\lambda , u) \to \lambda . u
    \end{align*}
  \end{itemize}

  $X$ verifies the following rules :
  \begin{itemize} \itemsep -4pt
  \item[-] $\forall u, v \in X$, $u + v = v + u$ (associative).
  \item[-] $\forall u, v, w \in X$, $ u + ( v + w) = ( u + v ) + w $.
  \item[-] There exist a neutral element $0_X$, such as $0_X \in X$, $ u + 0_X = u, \forall u \in X$
  \item[-] Each $u \in X$ has a symetric element $u' \in X$, such as $ u' + u = 0_X$ : $u'= -u$
  \item[-] $\lambda.(\mu.u) = (\lambda \mu). u$, $\forall \lambda, \mu \in \kappa, u \in X$
  \item[-] $\lambda.(u+v) = \lambda.u + \lambda.v$, $\forall \lambda \in \kappa$ and $u, v \in X $
  \item[-] $ (\lambda + \mu).u = \lambda.u + \mu.u$, $\forall \lambda, \mu \in \kappa$ and $u \in X$  
  \end{itemize} 
\end{definition}

In quantum mechanics, we shall use a similar concept of such vectorial spaces, deprived of concepts
such as direction.. 
\medskip
\begin{definition}
  Let $\varkappa$ be a linear vector space, and $\ket{1},\ket{2}, \ket{3},...\ket{V}..$ be elements of such
  vector space.\\
  We will see that such elements obeys laws similar to such of a $\kappa$-vectorial space.
  \end{definition}

\clearpage

\textbf{\textcolor{red}{Draft :: All of the equations will be modified later on.}}

Linear indepenence of a series of vectors :

\begin{equation}
  a_1 \ket{1} + a_2\ket{2} + a_3 \ket{3} + ... + a_n \ket{n} = \ket{0} = 0
\end{equation}
Can be expressed such as :
\begin{equation}
\sum^{n}_{i=1} a_i \ket{1} = \ket{0} 
\end{equation}

Expression of vectors in a \textbf{basis} :
\begin{equation}
\ket{V} = \sum^{n}_{i=1} \nu_i \ket{i} 
\end{equation}

Linear addition of 2 vectors :
\begin{equation}
\ket{V} = \sum^{n}_{i=1} \nu_i \ket{i} 
\end{equation}
and
\begin{equation}
\ket{W} = \sum^{n}_{i=1} \omega_i \ket{i} 
\end{equation}

\begin{equation}
\ket{V} + \ket{W}  = \sum^{n}_{i=1} \nu_i \ket{i} + \sum^{n}_{i=1} \omega_i \ket{i} = \sum^{n}_{i=1} (\omega_i + \nu_i) \ket{i}
 \end{equation}

Linear superposition :
\begin{align*}
    \braket{a W + b Z | V} = \braket{V | a W + b Z}^{*} 
  \end{align*}



\clearpage

\addcontentsline{toc}{section}{References}

\begin{thebibliography}{9}
\bibitem{Principles of Quantum Mechanics}
\textbf{Principles of Quantum Mechanics, second edition}\\
\texttt{Springer US, DOI 10.1007/978-1-4757-0576-8}\\
\texttt{Ramamurti Shankar, Yale University, New Haven, United States}\\

\end{thebibliography}



\end{document}
