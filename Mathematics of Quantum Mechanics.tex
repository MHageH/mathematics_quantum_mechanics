\documentclass[11pt]{article}
\renewcommand{\baselinestretch}{1.05}

\usepackage{amsmath,amsthm,verbatim,amssymb,amsfonts,amscd, graphicx}
\usepackage{graphics}

\usepackage{xcolor}

\usepackage[hidelinks]{hyperref}
\usepackage{parskip}

\usepackage{subcaption}
\usepackage{wrapfig}
\usepackage{multirow}

\setlength\tabcolsep{1mm}
\renewcommand\arraystretch{1.3}

\setlength\voffset{-1in}
\setlength\hoffset{-1in}
\setlength\topmargin{0.7874in}
\setlength\oddsidemargin{0.7874in}
\setlength\textheight{9.518099in}
\setlength\textwidth{6.6932993in}
\setlength\footskip{0.0cm}
\setlength\headheight{0cm}
\setlength\headsep{0cm}

\newcommand\textstyleEmphasis[1]{\textit{#1}}

\begin{document}

\title{\textbf{Mathematics of Quantum Mechanics} \\ Draft version - Barebone}
\author{Mohamed Hage Hassan}
\date{8 May 2017}
\maketitle
\thispagestyle{empty}

\begin{abstract}
     
\end{abstract}

\vskip 9.5cm
\begin{center} \textbf{Institut Polytechnique de Grenoble} \end{center}

\clearpage
\tableofcontents

\clearpage

\section{Introduction}

This text is largely inspired from professor Ramamurti Shankar's Principles of Quantum Mechanics\cite{Principles of Quantum Mechanics}.

\section{Linear Vector Spaces}

\clearpage

\addcontentsline{toc}{section}{References}

\begin{thebibliography}{9}
\bibitem{Principles of Quantum Mechanics}
\textbf{Principles of Quantum Mechanics, second edition}\\
\texttt{Springer US, DOI 10.1007/978-1-4757-0576-8}\\
\texttt{Ramamurti Shankar, Yale University, New Haven, United States}\\

\end{thebibliography}



\end{document}
